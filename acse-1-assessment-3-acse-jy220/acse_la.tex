%% Generated by Sphinx.
\def\sphinxdocclass{report}
\documentclass[letterpaper,10pt,english]{sphinxmanual}
\ifdefined\pdfpxdimen
   \let\sphinxpxdimen\pdfpxdimen\else\newdimen\sphinxpxdimen
\fi \sphinxpxdimen=.75bp\relax

\PassOptionsToPackage{warn}{textcomp}
\usepackage[utf8]{inputenc}
\ifdefined\DeclareUnicodeCharacter
% support both utf8 and utf8x syntaxes
  \ifdefined\DeclareUnicodeCharacterAsOptional
    \def\sphinxDUC#1{\DeclareUnicodeCharacter{"#1}}
  \else
    \let\sphinxDUC\DeclareUnicodeCharacter
  \fi
  \sphinxDUC{00A0}{\nobreakspace}
  \sphinxDUC{2500}{\sphinxunichar{2500}}
  \sphinxDUC{2502}{\sphinxunichar{2502}}
  \sphinxDUC{2514}{\sphinxunichar{2514}}
  \sphinxDUC{251C}{\sphinxunichar{251C}}
  \sphinxDUC{2572}{\textbackslash}
\fi
\usepackage{cmap}
\usepackage[T1]{fontenc}
\usepackage{amsmath,amssymb,amstext}
\usepackage{babel}



\usepackage{times}
\expandafter\ifx\csname T@LGR\endcsname\relax
\else
% LGR was declared as font encoding
  \substitutefont{LGR}{\rmdefault}{cmr}
  \substitutefont{LGR}{\sfdefault}{cmss}
  \substitutefont{LGR}{\ttdefault}{cmtt}
\fi
\expandafter\ifx\csname T@X2\endcsname\relax
  \expandafter\ifx\csname T@T2A\endcsname\relax
  \else
  % T2A was declared as font encoding
    \substitutefont{T2A}{\rmdefault}{cmr}
    \substitutefont{T2A}{\sfdefault}{cmss}
    \substitutefont{T2A}{\ttdefault}{cmtt}
  \fi
\else
% X2 was declared as font encoding
  \substitutefont{X2}{\rmdefault}{cmr}
  \substitutefont{X2}{\sfdefault}{cmss}
  \substitutefont{X2}{\ttdefault}{cmtt}
\fi


\usepackage[Bjarne]{fncychap}
\usepackage{sphinx}

\fvset{fontsize=\small}
\usepackage{geometry}


% Include hyperref last.
\usepackage{hyperref}
% Fix anchor placement for figures with captions.
\usepackage{hypcap}% it must be loaded after hyperref.
% Set up styles of URL: it should be placed after hyperref.
\urlstyle{same}


\usepackage{sphinxmessages}




\title{ACSE\_la}
\date{Oct 30, 2020}
\release{}
\author{unknown}
\newcommand{\sphinxlogo}{\vbox{}}
\renewcommand{\releasename}{}
\makeindex
\begin{document}

\pagestyle{empty}
\sphinxmaketitle
\pagestyle{plain}
\sphinxtableofcontents
\pagestyle{normal}
\phantomsection\label{\detokenize{index::doc}}



\chapter{A Gaussian Elimination routine}
\label{\detokenize{index:a-gaussian-elimination-routine}}
This package implements Gaussian elimination %
\begin{footnote}[1]\sphinxAtStartFootnote
\sphinxurl{https://mathworld.wolfram.com/GaussianElimination.html}
%
\end{footnote} for \sphinxcode{\sphinxupquote{numpy.ndarray}} objects, along with hand\sphinxhyphen{}written matrix multiplication and a hand written Bareiss Algorithm %
\begin{footnote}[2]\sphinxAtStartFootnote
\sphinxurl{http://informatika.stei.itb.ac.id/~rinaldi.munir/Matdis/2016-2017/Makalah2016/Makalah-Matdis-2016-051.pdf}
%
\end{footnote} for computing determinants.

See {\hyperref[\detokenize{index:acse_la.gauss}]{\sphinxcrossref{\sphinxcode{\sphinxupquote{acse\_la.gauss()}}}}}, \sphinxcode{\sphinxupquote{acse\_la.gauss.matmul()}} and \sphinxcode{\sphinxupquote{acse\_la.det.det()}} for more information.

\phantomsection\label{\detokenize{index:module-acse_la}}\index{module@\spxentry{module}!acse\_la@\spxentry{acse\_la}}\index{acse\_la@\spxentry{acse\_la}!module@\spxentry{module}}\index{gauss() (in module acse\_la)@\spxentry{gauss()}\spxextra{in module acse\_la}}

\begin{fulllineitems}
\phantomsection\label{\detokenize{index:acse_la.gauss}}\pysiglinewithargsret{\sphinxcode{\sphinxupquote{acse\_la.}}\sphinxbfcode{\sphinxupquote{gauss}}}{\emph{\DUrole{n}{a}}, \emph{\DUrole{n}{b}}}{}
Given two matrices, \sphinxtitleref{a} and \sphinxtitleref{b}, with \sphinxtitleref{a} square, the determinant
of \sphinxtitleref{a} and a matrix \sphinxtitleref{x} such that a*x = b are returned.
If \sphinxtitleref{b} is the identity, then \sphinxtitleref{x} is the inverse of \sphinxtitleref{a}.
\begin{quote}\begin{description}
\item[{Parameters}] \leavevmode\begin{itemize}
\item {} 
\sphinxstyleliteralstrong{\sphinxupquote{a}} (\sphinxstyleliteralemphasis{\sphinxupquote{np.array}}\sphinxstyleliteralemphasis{\sphinxupquote{ or }}\sphinxstyleliteralemphasis{\sphinxupquote{list of lists}}) \textendash{} ‘n x n’ array

\item {} 
\sphinxstyleliteralstrong{\sphinxupquote{b}} (\sphinxstyleliteralemphasis{\sphinxupquote{np. array}}\sphinxstyleliteralemphasis{\sphinxupquote{ or }}\sphinxstyleliteralemphasis{\sphinxupquote{list of lists}}) \textendash{} ‘m x n’ array

\end{itemize}

\end{description}\end{quote}
\subsubsection*{Examples}

\begin{sphinxVerbatim}[commandchars=\\\{\}]
\PYG{g+gp}{\PYGZgt{}\PYGZgt{}\PYGZgt{} }\PYG{n}{a} \PYG{o}{=} \PYG{p}{[}\PYG{p}{[}\PYG{l+m+mi}{2}\PYG{p}{,} \PYG{l+m+mi}{0}\PYG{p}{,} \PYG{o}{\PYGZhy{}}\PYG{l+m+mi}{1}\PYG{p}{]}\PYG{p}{,} \PYG{p}{[}\PYG{l+m+mi}{0}\PYG{p}{,} \PYG{l+m+mi}{5}\PYG{p}{,} \PYG{l+m+mi}{6}\PYG{p}{]}\PYG{p}{,} \PYG{p}{[}\PYG{l+m+mi}{0}\PYG{p}{,} \PYG{o}{\PYGZhy{}}\PYG{l+m+mi}{1}\PYG{p}{,} \PYG{l+m+mi}{1}\PYG{p}{]}\PYG{p}{]}
\PYG{g+gp}{\PYGZgt{}\PYGZgt{}\PYGZgt{} }\PYG{n}{b} \PYG{o}{=} \PYG{p}{[}\PYG{p}{[}\PYG{l+m+mi}{2}\PYG{p}{]}\PYG{p}{,} \PYG{p}{[}\PYG{l+m+mi}{1}\PYG{p}{]}\PYG{p}{,} \PYG{p}{[}\PYG{l+m+mi}{2}\PYG{p}{]}\PYG{p}{]}
\PYG{g+gp}{\PYGZgt{}\PYGZgt{}\PYGZgt{} }\PYG{n}{det}\PYG{p}{,} \PYG{n}{x} \PYG{o}{=} \PYG{n}{gauss}\PYG{p}{(}\PYG{n}{a}\PYG{p}{,} \PYG{n}{b}\PYG{p}{)}
\PYG{g+gp}{\PYGZgt{}\PYGZgt{}\PYGZgt{} }\PYG{n}{det}
\PYG{g+go}{22.0}
\PYG{g+gp}{\PYGZgt{}\PYGZgt{}\PYGZgt{} }\PYG{n}{x}
\PYG{g+go}{[[1.5], [\PYGZhy{}1.0], [1.0]]}
\PYG{g+gp}{\PYGZgt{}\PYGZgt{}\PYGZgt{} }\PYG{n}{A} \PYG{o}{=} \PYG{p}{[}\PYG{p}{[}\PYG{l+m+mi}{1}\PYG{p}{,} \PYG{l+m+mi}{0}\PYG{p}{,} \PYG{o}{\PYGZhy{}}\PYG{l+m+mi}{1}\PYG{p}{]}\PYG{p}{,} \PYG{p}{[}\PYG{o}{\PYGZhy{}}\PYG{l+m+mi}{2}\PYG{p}{,} \PYG{l+m+mi}{3}\PYG{p}{,} \PYG{l+m+mi}{0}\PYG{p}{]}\PYG{p}{,} \PYG{p}{[}\PYG{l+m+mi}{1}\PYG{p}{,} \PYG{o}{\PYGZhy{}}\PYG{l+m+mi}{3}\PYG{p}{,} \PYG{l+m+mi}{2}\PYG{p}{]}\PYG{p}{]}
\PYG{g+gp}{\PYGZgt{}\PYGZgt{}\PYGZgt{} }\PYG{n}{I} \PYG{o}{=} \PYG{p}{[}\PYG{p}{[}\PYG{l+m+mi}{1}\PYG{p}{,} \PYG{l+m+mi}{0}\PYG{p}{,} \PYG{l+m+mi}{0}\PYG{p}{]}\PYG{p}{,} \PYG{p}{[}\PYG{l+m+mi}{0}\PYG{p}{,} \PYG{l+m+mi}{1}\PYG{p}{,} \PYG{l+m+mi}{0}\PYG{p}{]}\PYG{p}{,} \PYG{p}{[}\PYG{l+m+mi}{0}\PYG{p}{,} \PYG{l+m+mi}{0}\PYG{p}{,} \PYG{l+m+mi}{1}\PYG{p}{]}\PYG{p}{]}
\PYG{g+gp}{\PYGZgt{}\PYGZgt{}\PYGZgt{} }\PYG{n}{Det}\PYG{p}{,} \PYG{n}{Ainv} \PYG{o}{=} \PYG{n}{gauss}\PYG{p}{(}\PYG{n}{A}\PYG{p}{,} \PYG{n}{I}\PYG{p}{)}
\PYG{g+gp}{\PYGZgt{}\PYGZgt{}\PYGZgt{} }\PYG{n}{Det}
\PYG{g+go}{3.0}
\PYG{g+gp}{\PYGZgt{}\PYGZgt{}\PYGZgt{} }\PYG{n}{Ainv} 
\PYG{g+go}{[[2.0, 1.0, 1.0],}
\PYG{g+go}{[1.3333333333333333, 1.0, 0.6666666666666666],}
\PYG{g+go}{[1.0, 1.0, 1.0]]}
\end{sphinxVerbatim}
\subsubsection*{Notes}

See \sphinxurl{https://en.wikipedia.org/wiki/Gaussian\_elimination} for further details.

\end{fulllineitems}



\begin{fulllineitems}
\pysiglinewithargsret{\sphinxcode{\sphinxupquote{acse\_la.gauss.}}\sphinxbfcode{\sphinxupquote{matmul}}}{\emph{\DUrole{n}{a}}, \emph{\DUrole{n}{b}}}{}
Given \sphinxtitleref{a} an n x m matrix and \sphinxtitleref{b} an m x l matrix, the product of a and b
is returned, as an n x l matrix.
\begin{quote}\begin{description}
\item[{Parameters}] \leavevmode\begin{itemize}
\item {} 
\sphinxstyleliteralstrong{\sphinxupquote{a}} (\sphinxstyleliteralemphasis{\sphinxupquote{np.array}}\sphinxstyleliteralemphasis{\sphinxupquote{ or }}\sphinxstyleliteralemphasis{\sphinxupquote{list of lists}}) \textendash{} ‘n x m’ array

\item {} 
\sphinxstyleliteralstrong{\sphinxupquote{b}} (\sphinxstyleliteralemphasis{\sphinxupquote{np. array}}\sphinxstyleliteralemphasis{\sphinxupquote{ or }}\sphinxstyleliteralemphasis{\sphinxupquote{list of lists}}) \textendash{} ‘m x l’ array

\end{itemize}

\end{description}\end{quote}
\subsubsection*{Examples}

\begin{sphinxVerbatim}[commandchars=\\\{\}]
\PYG{g+gp}{\PYGZgt{}\PYGZgt{}\PYGZgt{} }\PYG{n}{a} \PYG{o}{=} \PYG{p}{[}\PYG{p}{[}\PYG{l+m+mi}{1}\PYG{p}{,} \PYG{l+m+mi}{2}\PYG{p}{]}\PYG{p}{,} \PYG{p}{[}\PYG{l+m+mi}{3}\PYG{p}{,} \PYG{l+m+mi}{4}\PYG{p}{]}\PYG{p}{]}
\PYG{g+gp}{\PYGZgt{}\PYGZgt{}\PYGZgt{} }\PYG{n}{b} \PYG{o}{=} \PYG{p}{[}\PYG{p}{[}\PYG{l+m+mi}{5}\PYG{p}{]}\PYG{p}{,}\PYG{p}{[}\PYG{l+m+mi}{6}\PYG{p}{]}\PYG{p}{]}
\PYG{g+gp}{\PYGZgt{}\PYGZgt{}\PYGZgt{} }\PYG{n}{mul\PYGZus{}1} \PYG{o}{=} \PYG{n}{matmul}\PYG{p}{(}\PYG{n}{a}\PYG{p}{,} \PYG{n}{b}\PYG{p}{)}
\PYG{g+gp}{\PYGZgt{}\PYGZgt{}\PYGZgt{} }\PYG{n}{mul\PYGZus{}1}
\PYG{g+go}{[[17], [39]]}
\PYG{g+gp}{\PYGZgt{}\PYGZgt{}\PYGZgt{} }\PYG{n}{c} \PYG{o}{=} \PYG{p}{[}\PYG{p}{[}\PYG{l+m+mi}{5}\PYG{p}{,} \PYG{l+m+mi}{1}\PYG{p}{]}\PYG{p}{,} \PYG{p}{[}\PYG{l+m+mi}{6}\PYG{p}{,} \PYG{l+m+mi}{2}\PYG{p}{]}\PYG{p}{]}
\PYG{g+gp}{\PYGZgt{}\PYGZgt{}\PYGZgt{} }\PYG{n}{mul\PYGZus{}2} \PYG{o}{=} \PYG{n}{matmul}\PYG{p}{(}\PYG{n}{a}\PYG{p}{,} \PYG{n}{c}\PYG{p}{)}
\PYG{g+gp}{\PYGZgt{}\PYGZgt{}\PYGZgt{} }\PYG{n}{mul\PYGZus{}2}
\PYG{g+go}{[[17, 5], [39, 11]]}
\end{sphinxVerbatim}

\end{fulllineitems}



\begin{fulllineitems}
\pysiglinewithargsret{\sphinxcode{\sphinxupquote{acse\_la.gauss.}}\sphinxbfcode{\sphinxupquote{zeromat}}}{\emph{\DUrole{n}{p}}, \emph{\DUrole{n}{q}}}{}
Create an p x q matrix with all its entries be 0.
\begin{quote}\begin{description}
\item[{Parameters}] \leavevmode\begin{itemize}
\item {} 
\sphinxstyleliteralstrong{\sphinxupquote{p}} (\sphinxstyleliteralemphasis{\sphinxupquote{integer}}) \textendash{} 

\item {} 
\sphinxstyleliteralstrong{\sphinxupquote{q}} (\sphinxstyleliteralemphasis{\sphinxupquote{integer}}) \textendash{} 

\end{itemize}

\end{description}\end{quote}
\subsubsection*{Examples}

\begin{sphinxVerbatim}[commandchars=\\\{\}]
\PYG{g+gp}{\PYGZgt{}\PYGZgt{}\PYGZgt{} }\PYG{n}{p} \PYG{o}{=} \PYG{l+m+mi}{5}
\PYG{g+gp}{\PYGZgt{}\PYGZgt{}\PYGZgt{} }\PYG{n}{q} \PYG{o}{=} \PYG{l+m+mi}{6}
\PYG{g+gp}{\PYGZgt{}\PYGZgt{}\PYGZgt{} }\PYG{n}{z\PYGZus{}mat} \PYG{o}{=} \PYG{n}{zeromat}\PYG{p}{(}\PYG{n}{p}\PYG{p}{,} \PYG{n}{q}\PYG{p}{)}
\PYG{g+gp}{\PYGZgt{}\PYGZgt{}\PYGZgt{} }\PYG{n}{z\PYGZus{}mat} 
\PYG{g+go}{[[0, 0, 0, 0, 0, 0],}
\PYG{g+go}{[0, 0, 0, 0, 0, 0],}
\PYG{g+go}{[0, 0, 0, 0, 0, 0],}
\PYG{g+go}{[0, 0, 0, 0, 0, 0],}
\PYG{g+go}{[0, 0, 0, 0, 0, 0]]}
\end{sphinxVerbatim}

\end{fulllineitems}



\begin{fulllineitems}
\pysiglinewithargsret{\sphinxcode{\sphinxupquote{acse\_la.det.}}\sphinxbfcode{\sphinxupquote{det}}}{\emph{\DUrole{n}{a}}}{}
An “Bareiss Algorithm” to compute the determinant of a square matrix a.
\begin{quote}\begin{description}
\item[{Parameters}] \leavevmode
\sphinxstyleliteralstrong{\sphinxupquote{a}} (\sphinxstyleliteralemphasis{\sphinxupquote{np.array}}\sphinxstyleliteralemphasis{\sphinxupquote{ or }}\sphinxstyleliteralemphasis{\sphinxupquote{list of lists}}) \textendash{} ‘n x n’ array

\end{description}\end{quote}
\subsubsection*{Notes}

See \sphinxurl{http://informatika.stei.itb.ac.id/~rinaldi.munir/Matdis/2016-2017/Makalah2016/Makalah-Matdis-2016-051.pdf} Page.4 for further details.

\end{fulllineitems}

\subsubsection*{References}


\renewcommand{\indexname}{Python Module Index}
\begin{sphinxtheindex}
\let\bigletter\sphinxstyleindexlettergroup
\bigletter{a}
\item\relax\sphinxstyleindexentry{acse\_la}\sphinxstyleindexpageref{index:\detokenize{module-acse_la}}
\end{sphinxtheindex}

\renewcommand{\indexname}{Index}
\printindex
\end{document}