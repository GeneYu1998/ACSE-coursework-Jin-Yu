\documentclass[a4paper,11pt]{article}

\usepackage{hyperref}
\usepackage{amsmath}
\usepackage{amsfonts}
\usepackage{amssymb}
\usepackage{eurosym}
\usepackage{graphicx}
\usepackage{subfig}

\newcommand{\rr}{\mathbb{R}}
\newcommand{\rrn}{\mathbb{R}^{n}}
\newcommand{\nn}{\mathbb{N}}
\newcommand{\qq}{\mathbb{Q}}
\newcommand{\zz}{\mathbb{Z}}
\newcommand{\cc}{\mathbb{C}}
\newcommand{\dd}{\mathrm{d}}

\newenvironment{amatrix}[1]{%
  \left(\begin{array}{@{}*{#1}{c}|c@{}}
}{%
  \end{array}\right)
}

\newenvironment{detmatrix}[1]{%
  \left|\begin{array}{@{}*{#1}{c}}
}{%
  \end{array}\right|
}

    \makeatletter
    \renewcommand*\env@matrix[1][*\c@MaxMatrixCols c]{%
      \hskip -\arraycolsep
      \let\@ifnextchar\new@ifnextchar
      \array{#1}}
    \makeatother
    
%definition for scalar product and norm
\newcommand{\scal}[2]{\left\langle #1,#2 \right\rangle}
\newcommand{\norm}[1]{\left\|#1\right\|}

\newcommand{\Ra}{\Rightarrow}

%definition of averages
\newcommand{\averageh}[1]{\left\langle#1\right\rangle}
\newcommand{\averaget}[1]{\overline{ #1}}

%definition calculus
\newcommand{\grad}{\operatorname{grad}}
\renewcommand{\div}{\operatorname{div}}
\newcommand{\rot}{\operatorname{rot}}

\newcommand{\ig}[1]{\includegraphics[keepaspectratio=true, width=.7\textwidth]{#1}}
\newcommand{\igw}[2]{\includegraphics[keepaspectratio=true, width=#2\textwidth]{#1}}


\newtheorem{example}{Example}
\newtheorem{definition}{Definition}
\newtheorem{method}{Method}
\newtheorem{theorem}{Theorem}
\newtheorem{remark}{Remark}
\begin{document}
\begin{center}
\LARGE{ACSE-1 2020/21 - Assessment 3}\\
\vspace{0.5cm}
\LARGE{Testing, debugging, CI \& optimization}\\
\end{center}
\vspace{1.5cm}
%
This file contains instructions for completing the assignment. See the \newline
README.md file located in the base folder of this repository for instructions
regarding setting up the software.

The assessment is based around debugging, adding tests, docstrings and CI for a
\href{https://en.wikipedia.org/wiki/Gaussian\_elimination}{Gaussian elimination} algorithm
and then developing and optimising an algorithm for computing the determinant of matrices.
\textit{\textbf{Note}} that you do not need to understand the details of how to implement
a Gaussian elimination algorithm to complete this assignment, however you will need
to understand how to multiply two matrices together and how to compute the
\href{https://en.wikipedia.org/wiki/Determinant}{Determinant} of a square matrix.
Both of these linear algebra operations are explained below before detailing the
assessment.

\section*{Matrix multiplication}

Let $A$ be an $n \times m$ matrix and $B$ be an $m \times l$ matrix. We define the product of $A$ and $B$ as the dot product/scalar product of each row of the matrix $A$ with each column of the matrix $B$, that is
\begin{equation}
\begin{split}
A \cdot B &:=\begin{pmatrix}
a_{11} & a_{12} & \dots & a_{1m} \\
a_{21} & a_{22} & \dots & a_{2m} \\
\vdots &  \vdots & \dots &\vdots\\
a_{n1} & a_{n2} & \dots & a_{nm} \\
\end{pmatrix}\cdot
\begin{pmatrix}
b_{11} & b_{12} & \dots & b_{1l} \\
b_{21} & b_{22} & \dots & b_{2l} \\
\vdots &  \vdots & \dots &\vdots\\
b_{m1} & b_{m2} & \dots & b_{ml} \\
\end{pmatrix}
\\ &:=
\begin{pmatrix}
\sum_{j=1}^m a_{1j}b_{j1} & \dots & \sum_{j=1}^m a_{1j}b_{jl} \\
\sum_{j=1}^m a_{2j}b_{j1} & \dots & \sum_{j=1}^m a_{2j}b_{jl} \\
\vdots & \dots & \vdots \\
\sum_{j=1}^m a_{nj}b_{j1} & \dots & \sum_{j=1}^m a_{nj}b_{jl} \\
\end{pmatrix}
\end{split}
\end{equation}
and hence the result is an $n \times l$ matrix. A matrix is said to be square if $n = m$.

\begin{example}
$$
\begin{pmatrix}
1 & 2 \\
3 & 4
\end{pmatrix}
\begin{pmatrix}
5 \\ 6
\end{pmatrix}
=\begin{pmatrix}
5 + 12 \\
15 + 24 
\end{pmatrix}
=\begin{pmatrix}
17\\39
\end{pmatrix}
$$
\end{example}

\begin{example}
$$
\begin{pmatrix}
1 & 2 \\
3 & 4
\end{pmatrix}
\begin{pmatrix}
5 & 1 \\ 6 & 2
\end{pmatrix}
=\begin{pmatrix}
5 + 12 & 1 + 4 \\
15 + 24 & 3 + 8
\end{pmatrix}
=\begin{pmatrix}
17 & 5 \\ 39 & 11
\end{pmatrix}
$$
\end{example}

\section*{Determinant}

Consider an $n\times n$ matrix $A$. Furthermore, denote $B_{ij}$ the $(n-1)\times(n-1)$ matrix obtained from $A$ by removing the $i$-th row and the $j-th$ column. Then, it holds true that
\begin{equation}
\det(A) = \sum_{j=1}^{n} (-1)^{i+j} a_{ij} \det(B_{ij})
\end{equation}
for any fixed $i$ (row expansion), and
\begin{equation}
\det(A) = \sum_{i=1}^{n} (-1)^{i+j} a_{ij} \det(B_{ij}) 
\end{equation}
for any fixed $ j$ (column expansion).
The term $(-1)^{i+j}$ can be found easily by thinking of a chessboard: 
\begin{equation*}\begin{pmatrix}
+ & - & + & \hdots  \\
- & + & - & \hdots \\
+& - & &  \\
& &\ddots  & \vdots \\
\hdots & \hdots &- &+ 
\end{pmatrix}
\end{equation*}
By subsequent application, the computation of a determinant is broken down into a computation of many $2\times 2$ determinants. 
\begin{example}
\begin{equation*}
\det\begin{pmatrix}
2 & 3 & 7 & 9 \\
0 & 0 & 2 & 4 \\
0 & 1 & 5 & 0 \\
0 & 0 & 0 & 3 
\end{pmatrix}
= 2\cdot \det\begin{pmatrix}
0 & 2 & 4 \\ 1 & 5 & 0 \\0 & 0 & 3
\end{pmatrix} = 2\cdot (-1) \cdot \det\begin{pmatrix}
2 & 4 \\ 0 & 3
\end{pmatrix}=2\cdot (-1) \cdot6=-12.
\end{equation*}
\end{example}
Note that we have chosen which row/column to expand to minimize our workload. This example demonstrates that Laplace's formula saves effort when expanding a row or column containing many zeros.

\section*{Assessment}

\begin{enumerate}
\item Currently, running \texttt{flake8} from the base folder of this repository will reveal
several errors. Running \texttt{pytest tests/} will also reveal that the single test
located in the file \texttt{tests/test\_gauss.py} fails -- this is due to two bugs, one bug with the matrix
multiplication algorithm and one bug due to the computation of the determinant. Further, attempting to build
the \texttt{sphinx documentation} located in \texttt{docs} will fail due to a couple of errors.
\begin{enumerate}
 \item Make the repository \texttt{PEP8} compliant (i.e. fix the \texttt{flake8} errors)
 \item Add docstrings to the functions \texttt{matmul} and \texttt{zeromat}
 \item To \texttt{test\_gauss.py}, add additional tests for \texttt{matmul} and \texttt{zeromat}.
 (Note that you'll need to make these functions visible to the test file). Your tests
 should make use of the parameterize decorator to test multiple inputs.
 In doing this, you'll notice that any suitable test for the \texttt{matmul} function fails. Debug
 this function such that your test passes.
 \item With \texttt{matmul} fixed you'll notice that the \texttt{gauss} related test is still
 broken. Fix this bug
 so that \texttt{test\_gauss} passes and utilising the parameterize decorator add further (at least one) set of test inputs. \textbf{Note:} A properly working \texttt{gauss} function should be
 able to correctly return non integer determinants and this functionality should be tested.
 \item Next, add an additional file in tests called \newline
 \texttt{test\_docstrings.py} that
 tests the docstring tests in each of the three functions present in \texttt{gauss.py}.
 \item Finally for part 1, fix the bugs present in \texttt{docs/conf.py}. When fixed, use \texttt{sphinx}
 to build the associated \texttt{html} files and compile these into a \texttt{pdf} file named
 \texttt{ACSE\_la.pdf} which should be located in the \texttt{docs} folder. This documentation
 should be updated to reflect the final state of your repository.
\end{enumerate}
[40 marks]
\item The next task is to add some CI in the form of Github Actions workflows. These workflows should
be placed within the repository in the \texttt{.github/workflows} folder.
\begin{enumerate}
 \item Create a workflow that checks the workflow is PEP8 compliant. The workflow should trigger when
 (at the very least) a push is made the main branch.
 \item Create a workflow that runs \texttt{pytest} on all test files present within the \texttt{tests/}
 folder. The workflow should execute the tests on the following operating systems: (i) Ubuntu 20.04,
 (ii) MacOS 11.0 \& (iii) Windows Server 2019. (You may utilize the default \texttt{Python3} distribution
 available on those operating systems).
\end{enumerate}
[25 marks]
\item If we simply wish to compute the determinant of a matrix (e.g. \texttt{det = gauss(A, I)}),
clearly our current algorithm is not optimal, especially when the matrix becomes large
(~$10,000 \times 10,000$). Lets see how bad it is and if we can do better.
\begin{enumerate}
 \item Within the \texttt{scripts/} folder add a file called \texttt{det\_timings.py}. This script should,
 for many ($\approx10$) square matrices of increasing size ($2 \times 2 \Rightarrow 10,000 \times 10,000$), compute
 the time taken by the \texttt{gauss} algorithm
 to compute the determinant of these matrices. Additionally, for each of these matrices compute the time taken by
 \texttt{numpy.linalg.det} to calculate the determinant. Timing results should be
 written automatically by the script to a file named \texttt{timings.txt} in the \texttt{results/} folder with the
 formatting illustrated in Table \ref{timings_table}.
 \item Add a new workflow that, using a single operating system of your choice, executes the
 script \texttt{det\_timings.py}, commits the new results (i.e. the new \texttt{timings.txt} produced)
 and pushes them to your github repository.
 \item In the \texttt{acse\_la} folder add a new file \texttt{det.py} that contains a function (that you will write) named
 \texttt{det}. This function should be your own algorithm to compute \textit{only} the determinant
 of a single square matrix that's passed to it. How does your implementation compare to that of \texttt{gauss}
 and \texttt{numpy.linalg.det}? Have your script \texttt{det\_timings.py} also compute the timing of your
 \texttt{det} algorithm and add your results as a third column in \texttt{timings.txt}
\end{enumerate}
[35 marks]
\end{enumerate}

\begin{table}
\begin{center}
\begin{tabular}{ c c c }
 2 & 0.001 & 0.001 \\ 
 4 & 0.02 & 0.01 \\  
 $\vdots$ & $\vdots$ & $\vdots$ \\
 10000 & 5.0 & 1.0
\end{tabular}
\end{center}
\caption{\label{timings_table}Example formatting of the results table. The first column represents
the size of the matrix along one of its axes. The second column represents the corresponding timing
in a suitable unit of the \texttt{gauss} algorithm and the final column that of \texttt{numpy.linalg.det}.}
\end{table}

Notes:
\begin{itemize}
 \item Remember to ensure that the final version of your repository is \texttt{PEP8} compliant.
 \item Keep the sphinx documentation up to data.
 \item Additionally, ensure that your \texttt{requirements.txt} file has been updated
 appropriately to reflect any new dependencies you've added.
 \item For part 3(b) you have have your script and workflow overwrite the existing
 \texttt{timings.txt} file \textit{or} create a new file of the form
 \texttt{timings\_\textit{\{some identifier\}}.txt} and commit that upon each execution.
\end{itemize}

\end{document}
